\documentclass{eceasst}
% This is an empty ECEASST article that can be used as a template
% by authors.
% Just uncomment the appropriate frontmatter commands and provide
% the parameters.

% Required packages
% =================
% Your \usepackage commands go here.

% Volume frontmatter
% ==================
%\input{frontmatter}

% Volume frontmatter for AVoCS 2014
% =====================================
\volume{XXX}{2014} % Volume number and year
\volumetitle{% Title of the volume (optional)
Proceedings of the\\
14th International Workshop on\\
Automated Verification of Critical Systems
(AVoCS 2014)}
\volumeshort{% Short title of the volume (optional)
Proc.\ AVoCS 2014}
\guesteds{% Multiple guest editors
Marieke Huisman, Jaco van de Pol}
\title{No-Test Classes in C through Restricted Types} % Title of the article
%\short{} % Short title of the article (optional)
\author{% Authors and references to addresses
Dave Donaghy\autref{1} and 
Tom Crick\autref{2}}
\institute{% Institutes with labels
\autlabel{1} \email{dave.donaghy@hp.com}\\
HP Bristol, UK\par
\autlabel{2} \email{tcrick@cardiffmet.ac.uk}\\
Department of Computing\\
Cardiff Metropolitan University, UK}
\abstract{% Abstract of the article
Object-oriented programming (OOP) languages allow for the creation of
rich new types through, for example, the \begin{tt}class\end{tt}
mechanism found in C++ and Python (among others).

These techniques, while certainly rich in the functionality they
provide, additionally require users to develop and test new types;
while resulting software can be elegant and easy to understand (and
indeed these were some of the aspirations behind the OOP paradigm),
there is a cost associated to the addition of the new code required to
implement such new types. Such a cost will typically be at least
linear in the number of new types introduced.

One potential alternative to the creation of new types through
\begin{em}extension\end{em} is the creation of new types through
\begin{em}restriction\end{em}; in appropriate circumstances, such types
can provide the same elegance and ease of understanding, but without a
corresponding linear development and maintenance cost.
}
\keywords{Compilers, Plug-ins, Verification, Restricted Types} % Keywords for the article

\begin{document}
\maketitle

% Main part of your article
% =========================
\section{Introduction}

Object-oriented programming (OOP) languages allow for the creation of
rich new types through, for example, the \begin{tt}class\end{tt}
mechanism found in C++ and Python. However, it might be possible to
obtain some of the gains of such techniques without the associated
overheads in cost.

\section{Development Cost of New Types}

In an object-oriented development environment, it can reasonably be
said that \begin{em}all\end{em} software is encapsulated as methods on various
types; indeed, Java, for example, requires that all executable code be
written as type methods, allowing for the notion that
\begin{tt}static\end{tt} methods are still a kind of type method.

At the very least, then, the development of new types has
\begin{em}some\end{em} cost (and in particular, some financial or
resource cost) associated to it.  While we do not intend to directly
measure this cost, a fair starting assumption might be that is linear
in the number of new types introduced.

\section{Restricted Types}
One potential alternative to the creation of new types through
\begin{em}extension\end{em} is the creation of new types through
\begin{em}restriction\end{em}; in appropriate circumstances, such types
can provide the same elegance and ease of understanding, but without a
corresponding linear development and maintenance cost.

As an example, consider an integer counter, intended to represent the
number of occurrences of a certain event: the operations one might like to have
on such an entity can be described as follows:

\begin{enumerate}
\item Create a new counter, with a value of zero.
\item Increment the counter by one.
\item Compare the value of the counter against a given integer.
\end{enumerate}

Note that we might want to describe such operations
\begin{em}explicitly\end{em}, with the assumption that all other
operations (for example, multiplying the counter by 8, or setting bits 2,
3 and 7), are disallowed.

One could clearly create such an object simply (and elegantly) in C++
or Java using a class construct, but the point here is that
creation of such a new type would involve new, deployable, testable
software with a non-trivial associated cost; a counter such as this is,
mathematically and naturally speaking, a special kind of integer, and
therefore we already have all the required software (built into the 
hardware and run-time environment) that we need. In particular, what
we \begin{em}really\end{em} need is a constraint: we must promise not
to use disallowed ``non-counter'' operations on counters.

\section{Open Questions}

We can ask the following questions to frame future work in this area:

\begin{enumerate}
\item What existing common (or indeed uncommon) types naturally present
themselves as \begin{em}restrictions\end{em} of existing types, either
built-in/primitive types or other existing types?
\item What amount of software is involved in the definition of those types?
\item (Harder) What financial cost has historically been involved in
the creation and maintenance of those types?
\item What proportion of that cost might be saved by new techniques for
developing restricted types?
\end{enumerate}


% Acknowledgements for colleagues, referees, ...
% ==============================================
%\begin{acknowledge}
%\end{acknowledge}

% Bibliography with BibTeX
% ========================
%\bibliographystyle{eceasst}
%\bibliography{}

\end{document}
